\chapter{Implementation} \label{chap:Imp}

% Chapter giving an overview of the implementation of the method in OpenFOAM

For this project, all implementation of the regularization term and divergence cleaning scheme was carried out in C++ using the open source computational fluid dynamics code OpenFOAM. A solver developed during a previous project at the institute, which implemented the fundamental TLES methodology, was used as a starting code. This solver already contained the necessary code for discretizing and solving Eq. (\ref{TFNS}) in the TFNS system for the filtered velocity, Eq. (\ref{tau}) for the temporal stress tensor, as well as computing the deconvoluted velocity field according to the TEDM and the mean filtered velocity field for comparison and initial validation of the new stabilization methods.

For a numerical discretization scheme using a timestep $\Delta t$, the filter width $T$ is naturally parameterized by defining a filter width ratio $r$ as:
\begin{equation} \label{r}
r=\frac{T}{\Delta t}.
\end{equation}

A similar quanitity can also be defined for the filter used in generating the regularization quantity $\tilde{u}$ by defining a regularization filter width ratio $\tilde{r}$ as:
\begin{equation}
\tilde{r}=\frac{\tilde{T}}{T}.
\end{equation}

This definition of the ratios implies the restriction $\tilde{r}>1$ for logical consistency with the regularization theory, while in theory any $r>0$ is permissible. However, as stated in section \ref{sec:TLES}, it is deisrable to have $T>\Delta t$ and so $r>1$ can be considered as an unenforced lower bound in most practical use cases.

As suggested in section \ref{sec:DC}, the DC method can be implemented in a computationally efficient manner quite easily in OpenFOAM by specifying an iterative solver for the Poisson Eq. (\ref{eq:poisson}) in the `fvSolution' file with a relatively non-stringent tolerance. For this project, the same solver type was used as for the regular filtered velocity field, but the tolerance was relaxed from $10^{-5}$ to $10^{-2}$ as this did not noticeably affect the size of the increased range of stability, but did greatly reduce the observed computational time.
