\chapter{Conclusion}

% Chapter giving an overview of the conclusions which can be drawn from the project

A stabilizing regularization term and divergence cleaning via the projection method have been implemented and integrated into a previously developed temporally filtered large-eddy simulation (TLES) solver in OpenFOAM. The stability and output of the base method under different model parameters was studied and used for comparison against tests performed using the newly coded features to determine any stabilizing or other effects on the performance of the method.

Divergence cleaning was able to be efficiently implemented using an iterative solver with a relaxed tolerance, and was found to provide a measurable but small increase in the stable range of the filter width, along with very minor changes in the shape of its velocity line profile. The increase in stability seems to indicate that some non-zero divergence is present in the fields of the base method, but also that this was not the root cause of the instability, or at least not the sole root cause, since instability remained for larger filter widths.

Regularization using the addition of a linear forcing term to the filtered Navier-Stokes momentum equation was found to be quite effective at stabilizing the solution. Any value of the filter width tested was able to be simulated successfully by setting the control parameter $\chi$ of the regularization sufficiently large. The evolution of the system towards a steady state was found to be greatly slowed by the regularization, but did produce a very similar velocity line profile when a recognizable feature was propagated to the same location in the geometry as in the non-regularized runs. A simulation using a time step for which no filter width was stable in the base method was also able to be stabilized by the regularization, and it was additionally found that starting the simulation with already time-evolved initial data allowed for successful simulation with much larger filter widths that had been unstable for uniform initial conditions. These two results, while needing additional investigation, suggest possibilities for mitigating the slow system evolution, either by using a larger time step, or by starting a simulation with a number of iterations using a stable parameter set in order to produce a more stable starting point for the full simulation using the desired, but potentially less stable, configuration of the method.
