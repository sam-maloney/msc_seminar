% ABSTRACT
% ***************************************************************************************************
% with horizontal line
\begin{tabular}{p{0.9\textwidth}}
\chapter*{Abstract}

Temporal large eddy simulation (TLES) is a promising method to reduce computational costs compared to direct numerical simulation for turbulent flows while maintaining a high degree of accuracy, and offers several potential advantages over conventional spatial LES. A TLES method termed the temporal exact deconvolution model has previously been developed and then implemented in OpenFAOM but found to be unstable for large filter widths. In this project, a regularization term was designed, implemented, and tested to investigate its potential for stabilization of the method. Concurrently, a divergence cleaning method using the projection scheme was also added to the solver to test if non-zero divergence might be a source of the instabilities. Demonstration of the base method and the new stabilization strategies was realized using the Pitz-Daily backwards-facing step geometry provided in OpenFOAM, and a variety of different parameter combinations and model configurations were analyzed for both stability and for deviation from a stable baseline simulation. It was found that divergence cleaning improved the stability of the method only slightly, while the regularization term was able to stabilize every configuration tested if the strength of the regularization was made sufficiently high, although this stabilization came at the cost of substantial slowdown in the time-evolution of the flows towards their statistically stationary final states.

\end{tabular}
