% PREAMBLE
% ***************************************************************************************************

\documentclass[smallheadings,headsepline,11pt,oneside,a4paper]{scrbook}

% Hier gibt man an, welche Art von Dokument man schreiben möchte.
% Möglichkeiten in {}: scrartcl, scrreprt, scrbook, aber auch: article, report, book

\usepackage[english]{babel} 		% changed to remove German
\usepackage[utf8]{inputenc} 		% teilt LaTeX die Texcodierung mit. Bei Windowssystemen: ansinew
\usepackage[T1]{fontenc} 		% ermöglicht die Silbentrennung von Wörtern mit Umlauten

\usepackage[usenames,dvipsnames]{color}

% PDF wird mit Lesezeichen (verlinktes Inhaltsverzeichnis) versehen (bei Betrachtung mit Acrobat Reader sichtbar)
\definecolor{farbelink}{rgb}{0,0,0}	% Definition der Linkfarbe für das PDF-File
\usepackage[pdftex,colorlinks=true,urlcolor=farbelink,linkcolor=farbelink,citecolor=farbelink]{hyperref}

\usepackage{lscape}

\usepackage{amssymb,amsmath}

\usepackage{geometry} 
\geometry{a4paper} 
\usepackage[parfill]{parskip}    		% Activate to begin paragraphs with an empty line
\usepackage[pdftex]{graphicx}
\usepackage{epstopdf}
\DeclareGraphicsRule{.tif}{png}{.png}{`convert #1 `dirname #1`/`basename #1 .tif`.png}
\usepackage[margin=10pt,font=small,labelfont=bf]{caption}	% Package for picture captions

\usepackage[usenames]{color}

\usepackage{listings} 							% Einfügen von Programmcode
\lstset{numbers=left, numberstyle=\tiny, numbersep=5pt} 	% mit Zeilennummerierung links
\lstset{language=[77]Fortran}						% Programming language: Fortran77


%text with 2.5cm on both sides (A4=210mm)
\setlength{\textwidth}{160mm}
\setlength{\evensidemargin}{0in}
\setlength{\oddsidemargin}{0in}
\setlength{\columnsep}{0.25in}
%
%text with 2.5cm on top and 2.5cm on bottom (A4=297mm)
\setlength{\textheight}{247mm}
\setlength{\topmargin}{0pt}
\setlength{\headsep}{25pt}
\setlength{\headheight}{0pt}


%\typearea{12} % Breite des bedruckten Bereiches vergrössern (funktioniert nur in \documentclass mit: scrreprt, scrartcl, scrbook)

\clubpenalty = 10000 		% schliesst Schusterjungen aus
\widowpenalty = 10000 		% schliesst Hurenkinder aus


\usepackage{fancyhdr}
\pagestyle{fancy}		% Selbst gestaltete Kopf-/Fusszeilen

%%% Fancy Header %%%%%%%%%%%%%%%%%%%%%%%%%%%%%%%%%%%%%%%%%%%%%%%%%%%%%%%%%%%%%%%%%%
% Fancy Header Style Options
\fancyhf{}
\fancyfoot{} 

\renewcommand{\chaptermark}[1]{         		% Lower Case Chapter marker style
\markboth{\chaptername\ \thechapter.\ #1}{}}	%
\renewcommand{\sectionmark}[1]{         		% Lower case Section marker style
\markright{\thesection.\ #1}}         			%
\fancyfoot[R]{\thepage}    				% Page number in the right on every page
\fancyhead[L]{\leftmark}       				% Chapter in the right on even pages
\fancyhead[L]{\rightmark}     				% Section in the left on odd pages
\renewcommand{\headrulewidth}{0.3pt}		% Width of head rule

\fancypagestyle{plain}{					%
\fancyhf{}							% clear all header and footer fields
\fancyfoot[R]{\thepage}    				% Page number in the right on every page
\fancyhead[L]{\leftmark}       				% Chapter in the right on even pages
\renewcommand{\headrulewidth}{0.3pt}}   		% Width of head rule


\setlength{\textheight}{21cm}				% Vergrössert die Höhe des bedruckbaren Bereichs


\begin{document}

\frontmatter						% Roman numeral page numbers

% TITLE PAGE
% ***************************************************************************************************
% use FS or HS to identify the semester, FS (Fruehling) for spring semester, HS (Herbst) for fall semester

\begin{titlepage}
\begin{center}
    \vspace*{1cm}
    {\huge \bfseries  Development of a regularization term in a TLES code in OpenFOAM\\ }
    \vspace{2cm}
    {\large 
	Samuel Maloney\\
	~\\
	Computational Science and Engineering Master\\
	\vspace{3.5cm}
	Seminar in Fluid Dynamics for CSE HS 2017\\
	~\\
	Institute of Fluid Dynamics\\
	ETH Zürich\\
    }

\vspace{\stretch{1}}



{\large
	Supervisor: Daniel Oberle\\[\baselineskip]
	Professor: Patrick Jenny
}
\end{center}

\vspace*{2cm} % a bit of space at the bottom of the page

\end{titlepage}

%\clearpage\null


% Insert a blank page after the title page
\begin{titlepage}
\thispagestyle{empty}
\newpage
\mbox{}
\end{titlepage}


% ABSTRACT
% ***************************************************************************************************
% ABSTRACT
% ***************************************************************************************************
% with horizontal line
\begin{tabular}{p{0.9\textwidth}}
\chapter*{Abstract}

Temporal large eddy simulation (TLES) is a promising method to reduce computational costs compared to direct numerical simulation for turbulent flows while maintaining a high degree of accuracy, and offers several potential advantages over conventional spatial LES. A TLES method termed the temporal exact deconvolution model has previously been developed and then implemented in OpenFAOM but found to be unstable for large filter widths. In this project, a regularization term was designed, implemented, and tested to investigate its potential for stabilization of the method. Concurrently, a divergence cleaning method using the projection scheme was also added to the solver to test if non-zero divergence might be a source of the instabilities. Demonstration of the base method and the new stabilization strategies was realized using the Pitz-Daily backwards-facing step geometry provided in OpenFOAM, and a variety of different parameter combinations and model configurations were analyzed for both stability and for deviation from a stable baseline simulation. It was found that divergence cleaning improved the stability of the method only slightly, while the regularization term was able to stabilize every configuration tested if the strength of the regularization was made sufficiently high, although this stabilization came at the cost of substantial slowdown in the time-evolution of the flows towards their statistically stationary final states.

\end{tabular}




% TABLE OF CONTENTS
% ***************************************************************************************************

\tableofcontents
% Dieser Befehl erstellt das Inhaltsverzeichnis. Damit die Seitenzahlen korrekt sind, muss das Dokument zweimal gesetzt werden!



% MAIN SECTION
% ***************************************************************************************************
% Folgende Befehle stehen für die Gliederung zur Verfügung: \chapter \section \subsection \subsubsection \paragraph
% Für Anführungszeichen: "` (CH-Tastatur: ZUERST: Shift-Tast und Taste 2, DANN: Shift-Taste und Taste ^)
% Für Schlusszeichen: "' (CH- Tastatur: ZUERST: Shift-Tast und Taste 2, DANN:  '-Taste, rechts neben der Null)
% Für Neuen Abschnitt: Eine Zeile leer lassen.


\mainmatter	% Arabic page numbers


%1. Introduction
%------------------------------------------------------------------------------------------------------------------------------
\chapter{Introduction}

%Chapter giving an introduction to the topic and the present work

DNS is computationally expensive, so go TLES!

An brief overview of some of the basic theory underpinning TLES is presented next, adapting from the work presented by Pruett \cite{Pruett2008} and Jenny \cite{Jenny2016}.

\section{TLES}

Since for TLES filtering is done during the course of the numerical experiments, the filtering operation must be \emph{causal}, ie. it must depend only upon the current and previous values of the quantity being filtered and not on any future values. Letting an overbar denote a time-filtered quantity and $T$ denote the characteristic filter width, one obtains the following causal time-filtering operation for some filter kernal $G(t,T)$:
\begin{equation}
\bar{f}(t,T)=\int_{-\infty}^{t}G(\tau -t;T)f(\tau)d\tau
\end{equation}

In this work only the exponential filter is discussed, namely:
\begin{equation}
G(t;T)=\frac{\exp(t/T)}{T} \longrightarrow \bar{f}(t,T)=\frac{1}{T}\int_{-\infty}^{t}\exp\left(\frac{\tau-t}{T}\right)f(\tau)d\tau
\end{equation}

which is a second order low-pass filter and has the transfer function:
\begin{equation}
H(\omega,T)=\frac{1}{1+iT\omega}
\end{equation}

Since the integral formulation would require storage of the quanitity at all previous time points in the simulation, the following equivalent differential form is used for implementation, as it can be integrated using standard time-marching schemes to update the filtered quantities at each step (where the explicit time-dependence of the quanitites is now dropped for convenience):
\begin{equation} \label{filter_diff}
\frac{\partial \bar{f}}{\partial t}=\frac{f-\bar{f}}{T}
\end{equation}

Applying such a filtering operation to the incompressible Navier-Stokes equations gives the following system for the evolution of temporally-filtered velocity fields, using the same initial condition as for the unfiltered fields $\bar{u}(0,x;T)=u(0,x)$:
\begin{equation}
\frac{\partial \bar{u}_j}{\partial x_j}=0
\end{equation}
\begin{equation} \label{TFNS}
\frac{\partial \bar{u}_i}{\partial t}+\frac{\partial (\bar{u}_i\bar{u}_j)}{\partial x_j}=-\frac{\partial \bar{p}}{\partial x_i}+\nu \frac{\partial^2 \bar{u}_i}{\partial x_j \partial x_j}-\frac{\partial \tau_{ij}}{\partial x_j}
\end{equation}

where $u$ is the fluid velocity, $p$ is the pressure, $\nu$ is the kinematic pressure, and subscripts $i,j\in\{1,2,3\}$ indicate the 3D cartesian direction. The element $\tau$ in the final term is referred to as the \emph{temporal stress tensor} and is defined as:
\begin{equation}
\tau_{ij}=\overline{u_i u}_j-\bar{u}_i\bar{u}_j
\end{equation}

Solving the temporally-filtered Navier-Stokes equations requires a residual-stress model to handle the unkown value $\overline{u_i u}_j$ and close the system. Stolz and Adams \cite{Stolz2001} dveloped an approximate deconvolution model (ADM) for for spatial LES which was then adapted for TLES by Pruett \cite{Pruett2008}. For this project, however, a new method termed the temporal exact deconvolution model TEDM created by Jenny \cite{Jenny2016} is used to provide the closure.

For the TEDM method, it is observed that by inserting the velocity field into the differential form of the filter given in Eq. (\ref{filter_diff}) the unfiltered field can be recovered as follows:
\begin{equation} \label{deconv}
u_i=\bar{u}_i+T\frac{\partial \bar{u}_i}{\partial t}
\end{equation}

The same Eq. (\ref{filter_diff}) can then be applied to the unknown quantity $\overline{u_i u}_j$ and the just obtained relation for the unfiltered field in Eq. (\ref{deconv}) can be inserted in the resulting expression to obtain the following:
\begin{equation}
\begin{split}
\frac{\partial{\overline{u_i u}_j}}{\partial t}&=\frac{u_i u_j-\overline{u_i u}_j}{T} \\
&=\frac{\left( \bar{u}_i+T\frac{\partial \bar{u}_i}{\partial t} \right) \left( \bar{u}_j+T\frac{\partial \bar{u}_j}{\partial t} \right)-\overline{u_i u}_j}{T} \\
&=\frac{\bar{u}_i \bar{u}_j-\overline{u_i u}_j}{T}+\bar{u}_i\frac{\partial \bar{u}_j}{\partial t}+\bar{u}_j\frac{\partial \bar{u}_i}{\partial t}+T\frac{\partial \bar{u}_i}{\partial t}\frac{\partial \bar{u}_j}{\partial t}
\end{split}
\end{equation}

This can then be rearranged to produce an equation for the time evolution of the temporal stress tensor $\tau$ containing only the known filtered velocities:
\begin{equation} \label{tau}
\frac{\partial \tau_{ij}}{\partial t}=-\frac{\tau_{ij}}{T}+T\frac{\partial \bar{u}_i}{\partial t}\frac{\partial \bar{u}_j}{\partial t}
\end{equation}

Suitable time integration schemes can then be used to solve Eqs. (\ref{TFNS}) and (\ref{tau}) in alternation to evolve the system in time.

\section{Regularization}

A regularization term based on work by \AA kervik et al. \cite{Akervik2006} was investigated as a potential means to improve the stability of the method, and is outlined here.

\section{Divergence Cleaning}

An implicit condition of the incompressible Navier-Stokes equations is that the divergence of the velocity field must be zero. However, numerical discretization schemes generally lead to non-zero values of the divergence which can be a potential source of isntability during simulations. To this end, explicit divergence cleaning (DC) using the projection scheme, following the procedure outlined in section 5 of the work by T\'oth \cite{Toth2000} for DC of magnetic fields, was also considered for possible stabilizing effects.

%Addtional Chapters

%2. Vorgehen
%------------------------------------------------------------------------------------------------------------------------------
%\include{........}

% Bibliography
%------------------------------------------------------------------------------------------------------------------------------
% Bibliography
%*****************************************************************************************************

% Bewtween \begin{thebibliography}{99} and \end{thebibliography} kann das Literaturverzeichnis erweitert werden. 
% The text in the braces directly after \bibitem can be used for citation with the \cite command.


\begin{thebibliography}{99}
\addcontentsline{toc}{chapter}{Bibliography}

\bibitem{Pruett2008} \textsc {C. Pruett,}  ``Temporal large-eddy simulation: theory and implementation,'' Theor. Comput. Fluid Dyn. \textbf{22}, 275 (2008).

\bibitem{Stolz2001} \textsc {S. Stolz, N.A. Adams, and L. Kleiser,}  ``An approximate deconvolution model for large-eddy simulation with application to incompressible wall-bounded flows,'' Phys. Fluids  \textbf{13}, 997 (2001).

\bibitem{Akervik2006} \textsc {A. \AA kervik, L Brandt, D. S. Henningson, J. H\oe pffner, O. Marxen, and P. Schlatter,}  ``Steady solutions of the Navier-Stokes equations by selective frequency damping,'' Phys. Fluids  \textbf{18}, 068102 (2006).

\bibitem{Jenny2016} \textsc {P. Jenny,}  ``Unsteady RANS closure,'' Unpublished (2016).

\bibitem{Toth2000} \textsc {G. T\'oth,}  ``The $\nabla \cdot B=0$ Constraint in Shock-Capturing Magnetohydrodynamics Codes,'' J. Comput. Fluids \textbf{161}, 605 (2000).

% Citation format for a book
%\bibitem{citation_reference} \textsc {Author,}  \emph{Book Title}, year, Publisher.
 
\end{thebibliography}


%Attachment
%------------------------------------------------------------------------------------------------------------------------------

%\appendix
%\renewcommand{\chaptermark}[1]{         		% Benutze "Anhang" statt "Kapitel" in den Kopfzeilen
%\markboth{Attachment\ \thechapter.\ #1}{}} 

%\include{Appendix}



%ABBILDUNGSVERZEICHNIS
% ***************************************************************************************************
%\clearpage
%\phantomsection
%\addcontentsline{toc}{chapter}{Abbildungsverzeichnis}
%\listoffigures

%TABELLENVERZEICHNIS
% ***************************************************************************************************
%\clearpage
%\phantomsection
%\addcontentsline{toc}{chapter}{Tabellenverzeichnis}
%\listoftables


\end{document}
