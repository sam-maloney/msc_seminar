% PREAMBLE
% ***************************************************************************************************

\documentclass[smallheadings,headsepline,11pt,oneside,a4paper]{scrbook}

% Hier gibt man an, welche Art von Dokument man schreiben möchte.
% Möglichkeiten in {}: scrartcl, scrreprt, scrbook, aber auch: article, report, book

\usepackage[english]{babel} 		% changed to remove German
\usepackage[utf8]{inputenc} 		% teilt LaTeX die Texcodierung mit. Bei Windowssystemen: ansinew
\usepackage[T1]{fontenc} 		% ermöglicht die Silbentrennung von Wörtern mit Umlauten

\usepackage[usenames,dvipsnames]{color}

% PDF wird mit Lesezeichen (verlinktes Inhaltsverzeichnis) versehen (bei Betrachtung mit Acrobat Reader sichtbar)
\definecolor{farbelink}{rgb}{0,0,0}	% Definition der Linkfarbe für das PDF-File
\usepackage[pdftex,colorlinks=true,urlcolor=farbelink,linkcolor=farbelink,citecolor=farbelink]{hyperref}

\usepackage{lscape}

\usepackage{amssymb,amsmath}

\usepackage{geometry} 
\geometry{a4paper} 
\usepackage[parfill]{parskip}    								% Activate to begin paragraphs with an empty line
\usepackage[pdftex]{graphicx}
\usepackage{epstopdf}
\DeclareGraphicsRule{.tif}{png}{.png}{`convert #1 `dirname #1`/`basename #1 .tif`.png}
\usepackage[margin=10pt,font=small,labelfont=bf]{caption}	% Package for picture captions

\usepackage[usenames]{color}

\usepackage{listings} 						% Einfügen von Programmcode
\lstset{numbers=left, numberstyle=\tiny, numbersep=5pt} 	% mit Zeilennummerierung links
\lstset{language=[77]Fortran}								%Programmiersprache: Fortran77


%text with 2.5cm on both sides (A4=210mm)
\setlength{\textwidth}{160mm}
\setlength{\evensidemargin}{0in}
\setlength{\oddsidemargin}{0in}
\setlength{\columnsep}{0.25in}
%
%text with 2.5cm on top and 2.5cm on bottom (A4=297mm)
\setlength{\textheight}{247mm}
\setlength{\topmargin}{0pt}
\setlength{\headsep}{25pt}
\setlength{\headheight}{0pt}


%\typearea{12} % Breite des bedruckten Bereiches vergrössern (funktioniert nur in \documentclass mit: scrreprt, scrartcl, scrbook)

\clubpenalty = 10000 		% schliesst Schusterjungen aus
\widowpenalty = 10000 		% schliesst Hurenkinder aus


\usepackage{fancyhdr}
\pagestyle{fancy}		% Selbst gestaltete Kopf-/Fusszeilen

%%% Fancy Header %%%%%%%%%%%%%%%%%%%%%%%%%%%%%%%%%%%%%%%%%%%%%%%%%%%%%%%%%%%%%%%%%%
% Fancy Header Style Options
\fancyhf{}
\fancyfoot{} 

\renewcommand{\chaptermark}[1]{         		% Lower Case Chapter marker style
\markboth{\chaptername\ \thechapter.\ #1}{}} 		%
\renewcommand{\sectionmark}[1]{         		% Lower case Section marker style
\markright{\thesection.\ #1}}         			%
\fancyfoot[R]{\thepage}    				% Page number in the right on every page
\fancyhead[L]{\leftmark}       				% Chapter in the right on even pages
\fancyhead[L]{\rightmark}     				% Section in the left on odd pages
\renewcommand{\headrulewidth}{0.3pt}   			% Width of head rule

\fancypagestyle{plain}{					%
\fancyhf{} 						% clear all header and footer fields
\fancyfoot[R]{\thepage}    				% Page number in the right on every page
\fancyhead[L]{\leftmark}       				% Chapter in the right on even pages
\renewcommand{\headrulewidth}{0.3pt}}   		% Width of head rule


\setlength{\textheight}{21cm}				% Vergrössert die Höhe des bedruckbaren Bereichs


\begin{document}

\frontmatter						% Roman numeral page numbers

% TITLE PAGE
% ***************************************************************************************************
% use FS or HS to identify the semester, FS (Fruehling) for spring semester, HS (Herbst) for fall semester

\begin{titlepage}
\begin{center}
    \vspace*{1cm}
    {\huge \bfseries  Development of a Regularization Term in a TLES Code in OpenFOAM\\ }
    \vspace{2cm}
    {\large 
	Samuel Maloney\\
	~\\
	Computational Science and Engineering Master\\
	\vspace{3.5cm}
	Seminar in Fluid Dynamics for CSE HS 2017\\
	~\\
	Institute of Fluid Dynamics\\
	ETH Zürich\\
    }

\vspace{\stretch{1}}



{\large
	Supervisor: Daniel Oberle\\[\baselineskip]
	Professor: Patrick Jenny
}
\end{center}

\vspace*{2cm} % a bit of space at the bottom of the page

\end{titlepage}

%\clearpage\null


% Insert a blank page after the title page
\begin{titlepage}
\thispagestyle{empty}
\newpage
\mbox{}
\end{titlepage}


% ABSTRACT
% ***************************************************************************************************
% ABSTRACT
% ***************************************************************************************************
% with horizontal line
\begin{tabular}{p{0.9\textwidth}}
\chapter*{Abstract}

Temporal large eddy simulation (TLES) is a promising method to reduce computational costs compared to direct numerical simulation for turbulent flows while maintaining a high degree of accuracy, and offers several potential advantages over conventional spatial LES. A TLES method termed the temporal exact deconvolution model has previously been developed and then implemented in OpenFAOM but found to be unstable for large filter widths. In this project, a regularization term was designed, implemented, and tested to investigate its potential for stabilization of the method. Concurrently, a divergence cleaning method using the projection scheme was also added to the solver to test if non-zero divergence might be a source of the instabilities. Demonstration of the base method and the new stabilization strategies was realized using the Pitz-Daily backwards-facing step geometry provided in OpenFOAM, and a variety of different parameter combinations and model configurations were analyzed for both stability and for deviation from a stable baseline simulation. It was found that divergence cleaning improved the stability of the method only slightly, while the regularization term was able to stabilize every configuration tested if the strength of the regularization was made sufficiently high, although this stabilization came at the cost of substantial slowdown in the time-evolution of the flows towards their statistically stationary final states.

\end{tabular}




% TABLE OF CONTENTS
% ***************************************************************************************************

\tableofcontents
% Dieser Befehl erstellt das Inhaltsverzeichnis. Damit die Seitenzahlen korrekt sind, muss das Dokument zweimal gesetzt werden!



% MAIN SECTION
% ***************************************************************************************************
% Folgende Befehle stehen für die Gliederung zur Verfügung: \chapter \section \subsection \subsubsection \paragraph
% Für Anführungszeichen: "` (CH-Tastatur: ZUERST: Shift-Tast und Taste 2, DANN: Shift-Taste und Taste ^)
% Für Schlusszeichen: "' (CH- Tastatur: ZUERST: Shift-Tast und Taste 2, DANN:  '-Taste, rechts neben der Null)
% Für Neuen Abschnitt: Eine Zeile leer lassen.


\mainmatter	% Arabic page numbers


%1. Introduction
%------------------------------------------------------------------------------------------------------------------------------
\chapter{Introduction}

%Chapter giving an introduction to the topic and the present work


%Addtional Chapters

%2. Vorgehen
%------------------------------------------------------------------------------------------------------------------------------
%\include{........}

%Bibliography
%------------------------------------------------------------------------------------------------------------------------------
% Bibliography
%*****************************************************************************************************

% Bewtween \begin{thebibliography}{99} and \end{thebibliography} kann das Literaturverzeichnis erweitert werden. 
% The text in the braces directly after \bibitem can be used for citation with the \cite command.


\begin{thebibliography}{99}
\addcontentsline{toc}{chapter}{Bibliography}

\bibitem{Pruett2008} \textsc {C. Pruett,}  ``Temporal large-eddy simulation: theory and implementation,'' Theor. Comput. Fluid Dyn. \textbf{22}, 275 (2008).

\bibitem{Jenny2016} \textsc {P. Jenny,}  ``Unsteady RANS closure,'' Unpublished (2016).

\bibitem{Stolz2001} \textsc {S. Stolz, N.A. Adams, and L. Kleiser,}  ``An approximate deconvolution model for large-eddy simulation with application to incompressible wall-bounded flows,'' Phys. Fluids  \textbf{13}, 997 (2001).

\bibitem{Akervik2006} \textsc {A. \AA kervik, L Brandt, D. S. Henningson, J. H\oe pffner, O. Marxen, and P. Schlatter,}  ``Steady solutions of the Navier-Stokes equations by selective frequency damping,'' Phys. Fluids  \textbf{18}, 068102 (2006).

\bibitem{Toth2000} \textsc {G. T\'oth,}  ``The $\nabla \cdot B=0$ Constraint in Shock-Capturing Magnetohydrodynamics Codes,'' J. Comput. Fluids \textbf{161}, 605 (2000).

% Citation format for a book
%\bibitem{citation_reference} \textsc {Author,}  \emph{Book Title}, year, Publisher.
 
\end{thebibliography}


%Attachment
%------------------------------------------------------------------------------------------------------------------------------
\appendix
\renewcommand{\chaptermark}[1]{         		% Benutze "Anhang" statt "Kapitel" in den Kopfzeilen
\markboth{Attachment\ \thechapter.\ #1}{}} 

\chapter{Appendix}

%Attachment




%ABBILDUNGSVERZEICHNIS
% ***************************************************************************************************
%\clearpage
%\phantomsection
%\addcontentsline{toc}{chapter}{Abbildungsverzeichnis}
%\listoffigures

%TABELLENVERZEICHNIS
% ***************************************************************************************************
%\clearpage
%\phantomsection
%\addcontentsline{toc}{chapter}{Tabellenverzeichnis}
%\listoftables


\end{document}
